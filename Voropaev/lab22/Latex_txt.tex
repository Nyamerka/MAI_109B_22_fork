\documentclass{article}
\usepackage{amsmath}
\usepackage{color,pxfonts,fix-cm}
\usepackage{latexsym}
\usepackage[mathletters]{ucs}
\DeclareUnicodeCharacter{8211}{\textendash}
\DeclareUnicodeCharacter{46}{\textperiodcentered}
\DeclareUnicodeCharacter{124}{\textbar}
\DeclareUnicodeCharacter{32}{$\ $}
\usepackage[T1]{fontenc}
\usepackage[utf8x]{inputenc}
\usepackage{pict2e}
\usepackage{wasysym}
\usepackage[english]{babel}
\usepackage{tikz}
\pagestyle{empty}
\usepackage[margin=0in,paperwidth=595pt,paperheight=842pt]{geometry}
\begin{document}
\definecolor{color_29791}{rgb}{0,0,0}
\begin{tikzpicture}[overlay]\path(0pt,0pt);\end{tikzpicture}
\begin{picture}(-5,0)(2.5,0)
\put(41.7,-786.7997){\fontsize{12.0239}{1}\usefont{T1}{cmr}{m}{n}\selectfont\color{color_29791} }
\put(276.659,-786.8598){\fontsize{12.0239}{1}\usefont{T1}{cmr}{m}{n}\selectfont\color{color_29791}39}
\put(41.7,-65.6543){\fontsize{12.0239}{1}\usefont{T1}{cmr}{m}{n}\selectfont\color{color_29791} }
\end{picture}
\begin{tikzpicture}[overlay]
\path(0pt,0pt);
\draw[color_29791,line width=0.998pt,line cap=round,line join=round]
(298.2pt, -62.65997pt) -- (315pt, -62.65997pt)
;
\end{tikzpicture}
\begin{picture}(-5,0)(2.5,0)
\put(311.94,-75.19971){\fontsize{12.007}{1}\usefont{T1}{cmr}{m}{n}\selectfont\color{color_29791}|v|}
\put(312.3002,-57.92157){\fontsize{12.007}{1}\usefont{T1}{cmr}{m}{n}\selectfont\color{color_29791}|h|}
\put(265.9772,-65.66608){\fontsize{12.007}{1}\usefont{T1}{cmr}{m}{n}\selectfont\color{color_29791}arcsin. }
\put(41.71046,-98.66138){\fontsize{12.007}{1}\usefont{T1}{cmr}{m}{n}\selectfont\color{color_29791} }
\put(41.7,-124.1597){\fontsize{12.0504}{1}\usefont{T1}{cmr}{m}{n}\selectfont\color{color_29791} Пример 4. Можно ли в векторных пространствах }
\put(334.74,-118.7597){\fontsize{10.0052}{1}\usefont{T1}{cmr}{m}{n}\selectfont\color{color_29791}2}
\put(325.5,-124.1597){\fontsize{11.9894}{1}\usefont{T1}{cmr}{m}{n}\selectfont\color{color_29791} (столбцов из двух действительных }
\put(41.70511,-145.7603){\fontsize{12.007}{1}\usefont{T2A}{cmr}{m}{n}\selectfont\color{color_29791}чисел) и }
\put(97.14,-150.0197){\fontsize{10.02}{1}\usefont{T1}{cmr}{m}{n}\selectfont\color{color_29791}2}
\put(91.68,-145.8197){\fontsize{12.0239}{1}\usefont{T1}{cmr}{m}{n}\selectfont\color{color_29791}P}
\put(104.52,-145.7597){\fontsize{12.007}{1}\usefont{T1}{cmr}{m}{n}\selectfont\color{color_29791} (многочленов степени не выше второй) задать скалярное произведение форму-}
\put(41.72339,-170.7223){\fontsize{12.007}{1}\usefont{T2A}{cmr}{m}{n}\selectfont\color{color_29791}лами }
\put(70.08,-170.7197){\fontsize{12.0504}{1}\usefont{T1}{cmr}{m}{n}\selectfont\color{color_29791}(1) или (2), приведенными в таблице? }
\put(41.7,-184.8797){\fontsize{4.98}{1}\usefont{T1}{cmr}{m}{n}\selectfont\color{color_29791} }
\put(47.88,-200.6597){\fontsize{12.0504}{1}\usefont{T2A}{cmr}{m}{n}\selectfont\color{color_29791}Пр-во Формула (1) Формула (2) }
\end{picture}
\begin{tikzpicture}[overlay]
\path(0pt,0pt);
\filldraw[color_29791][nonzero rule]
(41.22pt, -189.5pt) -- (41.7pt, -189.5pt)
 -- (41.7pt, -189.5pt)
 -- (41.7pt, -189.02pt)
 -- (41.7pt, -189.02pt)
 -- (41.22pt, -189.02pt) -- cycle
;
\filldraw[color_29791][nonzero rule]
(41.22pt, -189.5pt) -- (524.1pt, -189.5pt)
 -- (524.1pt, -189.5pt)
 -- (524.1pt, -189.02pt)
 -- (524.1pt, -189.02pt)
 -- (41.22pt, -189.02pt) -- cycle
;
\filldraw[color_29791][nonzero rule]
(523.62pt, -189.5pt) -- (524.1pt, -189.5pt)
 -- (524.1pt, -189.5pt)
 -- (524.1pt, -189.02pt)
 -- (524.1pt, -189.02pt)
 -- (523.62pt, -189.02pt) -- cycle
;
\filldraw[color_29791][nonzero rule]
(41.22pt, -210.2pt) -- (41.7pt, -210.2pt)
 -- (41.7pt, -210.2pt)
 -- (41.7pt, -189.5pt)
 -- (41.7pt, -189.5pt)
 -- (41.22pt, -189.5pt) -- cycle
;
\filldraw[color_29791][nonzero rule]
(86.58pt, -210.2pt) -- (87.06001pt, -210.2pt)
 -- (87.06001pt, -210.2pt)
 -- (87.06001pt, -189.5pt)
 -- (87.06001pt, -189.5pt)
 -- (86.58pt, -189.5pt) -- cycle
;
\filldraw[color_29791][nonzero rule]
(293.52pt, -210.2pt) -- (294pt, -210.2pt)
 -- (294pt, -210.2pt)
 -- (294pt, -189.5pt)
 -- (294pt, -189.5pt)
 -- (293.52pt, -189.5pt) -- cycle
;
\filldraw[color_29791][nonzero rule]
(523.62pt, -210.2pt) -- (524.1pt, -210.2pt)
 -- (524.1pt, -210.2pt)
 -- (524.1pt, -189.5pt)
 -- (524.1pt, -189.5pt)
 -- (523.62pt, -189.5pt) -- cycle
;
\end{tikzpicture}
\begin{picture}(-5,0)(2.5,0)
\put(66.18,-221.7197){\fontsize{10.0052}{1}\usefont{T1}{cmr}{m}{n}\selectfont\color{color_29791}2}
\put(56.94,-227.1197){\fontsize{11.9894}{1}\usefont{T1}{cmr}{m}{n}\selectfont\color{color_29791} }
\put(276.06,-226.8197){\fontsize{10.02}{1}\usefont{T1}{cmr}{m}{n}\selectfont\color{color_29791}22122111}
\put(252.42,-222.6197){\fontsize{12.007}{1}\usefont{T1}{cmr}{m}{n}\selectfont\color{color_29791}3224),(yxyxyxyxyx }
\put(492.4799,-226.5797){\fontsize{9.5684}{1}\usefont{T1}{cmr}{m}{n}\selectfont\color{color_29791}22122111}
\put(469.56,-222.5597){\fontsize{11.482}{1}\usefont{T1}{cmr}{m}{n}\selectfont\color{color_29791}3442),(yxyxyxyxyx}
\put(499.74,-222.2597){\fontsize{12.007}{1}\usefont{T1}{cmr}{m}{n}\selectfont\color{color_29791} }
\end{picture}
\begin{tikzpicture}[overlay]
\path(0pt,0pt);
\filldraw[color_29791][nonzero rule]
(41.22pt, -210.68pt) -- (524.1pt, -210.68pt)
 -- (524.1pt, -210.68pt)
 -- (524.1pt, -210.2pt)
 -- (524.1pt, -210.2pt)
 -- (41.22pt, -210.2pt) -- cycle
;
\filldraw[color_29791][nonzero rule]
(41.22pt, -236.6pt) -- (41.7pt, -236.6pt)
 -- (41.7pt, -236.6pt)
 -- (41.7pt, -210.68pt)
 -- (41.7pt, -210.68pt)
 -- (41.22pt, -210.68pt) -- cycle
;
\filldraw[color_29791][nonzero rule]
(86.58pt, -236.6pt) -- (87.06001pt, -236.6pt)
 -- (87.06001pt, -236.6pt)
 -- (87.06001pt, -210.68pt)
 -- (87.06001pt, -210.68pt)
 -- (86.58pt, -210.68pt) -- cycle
;
\filldraw[color_29791][nonzero rule]
(293.52pt, -236.6pt) -- (294pt, -236.6pt)
 -- (294pt, -236.6pt)
 -- (294pt, -210.68pt)
 -- (294pt, -210.68pt)
 -- (293.52pt, -210.68pt) -- cycle
;
\filldraw[color_29791][nonzero rule]
(523.62pt, -236.6pt) -- (524.1pt, -236.6pt)
 -- (524.1pt, -236.6pt)
 -- (524.1pt, -210.68pt)
 -- (524.1pt, -210.68pt)
 -- (523.62pt, -210.68pt) -- cycle
;
\end{tikzpicture}
\begin{picture}(-5,0)(2.5,0)
\put(64.2,-265.8197){\fontsize{10.02}{1}\usefont{T1}{cmr}{m}{n}\selectfont\color{color_29791}2}
\put(58.8,-261.6197){\fontsize{12.0239}{1}\usefont{T1}{cmr}{m}{n}\selectfont\color{color_29791}P }
\put(282.18,-261.9797){\fontsize{11.61}{1}\usefont{T1}{cmr}{m}{n}\selectfont\color{color_29791})0()0(2)1()1()1()1(),(qpqpqpqp}
\put(269.82,-261.4397){\fontsize{11.61}{1}\usefont{T1}{cmr}{m}{n}\selectfont\color{color_29791}}
\put(225.6,-261.9797){\fontsize{11.61}{1}\usefont{T1}{cmr}{m}{n}\selectfont\color{color_29791}}
\put(190.4449,-261.4456){\fontsize{11.61}{1}\usefont{T1}{cmr}{m}{n}\selectfont\color{color_29791}}
\put(174.4812,-261.9797){\fontsize{11.61}{1}\usefont{T1}{cmr}{m}{n}\selectfont\color{color_29791}}
\put(157.8673,-261.4456){\fontsize{11.61}{1}\usefont{T1}{cmr}{m}{n}\selectfont\color{color_29791}}
\put(122.5264,-261.9797){\fontsize{11.61}{1}\usefont{T1}{cmr}{m}{n}\selectfont\color{color_29791}}
\put(512.8199,-262.6396){\fontsize{11.7919}{1}\usefont{T1}{cmr}{m}{n}\selectfont\color{color_29791})1()1()]()()()([),(}
\put(340.92,-247.4597){\fontsize{9.8266}{1}\usefont{T1}{cmr}{m}{n}\selectfont\color{color_29791}1}
\put(343.8582,-279.3175){\fontsize{9.8266}{1}\usefont{T1}{cmr}{m}{n}\selectfont\color{color_29791}1}
\put(498.72,-262.6396){\fontsize{11.7919}{1}\usefont{T1}{cmr}{m}{n}\selectfont\color{color_29791}qpdxxqxpxqxpqp}
\put(435.6687,-261.9793){\fontsize{11.7919}{1}\usefont{T1}{cmr}{m}{n}\selectfont\color{color_29791}}
\put(396.0715,-262.6396){\fontsize{11.7919}{1}\usefont{T1}{cmr}{m}{n}\selectfont\color{color_29791}}
\put(340.8,-267.6797){\fontsize{19.6533}{1}\usefont{T1}{cmr}{m}{n}\selectfont\color{color_29791}}
\put(339.06,-279.3197){\fontsize{9.8266}{1}\usefont{T1}{cmr}{m}{n}\selectfont\color{color_29791}}
\end{picture}
\begin{tikzpicture}[overlay]
\path(0pt,0pt);
\filldraw[color_29791][nonzero rule]
(41.22pt, -237.08pt) -- (524.1pt, -237.08pt)
 -- (524.1pt, -237.08pt)
 -- (524.1pt, -236.6pt)
 -- (524.1pt, -236.6pt)
 -- (41.22pt, -236.6pt) -- cycle
;
\filldraw[color_29791][nonzero rule]
(41.22pt, -288.74pt) -- (41.7pt, -288.74pt)
 -- (41.7pt, -288.74pt)
 -- (41.7pt, -237.08pt)
 -- (41.7pt, -237.08pt)
 -- (41.22pt, -237.08pt) -- cycle
;
\filldraw[color_29791][nonzero rule]
(41.22pt, -288.74pt) -- (86.58pt, -288.74pt)
 -- (86.58pt, -288.74pt)
 -- (86.58pt, -288.26pt)
 -- (86.58pt, -288.26pt)
 -- (41.22pt, -288.26pt) -- cycle
;
\filldraw[color_29791][nonzero rule]
(86.58pt, -288.74pt) -- (87.06001pt, -288.74pt)
 -- (87.06001pt, -288.74pt)
 -- (87.06001pt, -237.08pt)
 -- (87.06001pt, -237.08pt)
 -- (86.58pt, -237.08pt) -- cycle
;
\filldraw[color_29791][nonzero rule]
(87.06pt, -288.74pt) -- (293.52pt, -288.74pt)
 -- (293.52pt, -288.74pt)
 -- (293.52pt, -288.26pt)
 -- (293.52pt, -288.26pt)
 -- (87.06pt, -288.26pt) -- cycle
;
\filldraw[color_29791][nonzero rule]
(293.52pt, -288.74pt) -- (294pt, -288.74pt)
 -- (294pt, -288.74pt)
 -- (294pt, -237.08pt)
 -- (294pt, -237.08pt)
 -- (293.52pt, -237.08pt) -- cycle
;
\filldraw[color_29791][nonzero rule]
(294pt, -288.74pt) -- (523.62pt, -288.74pt)
 -- (523.62pt, -288.74pt)
 -- (523.62pt, -288.26pt)
 -- (523.62pt, -288.26pt)
 -- (294pt, -288.26pt) -- cycle
;
\filldraw[color_29791][nonzero rule]
(523.62pt, -288.74pt) -- (524.1pt, -288.74pt)
 -- (524.1pt, -288.74pt)
 -- (524.1pt, -237.08pt)
 -- (524.1pt, -237.08pt)
 -- (523.62pt, -237.08pt) -- cycle
;
\filldraw[color_29791][nonzero rule]
(523.62pt, -288.74pt) -- (524.1pt, -288.74pt)
 -- (524.1pt, -288.74pt)
 -- (524.1pt, -288.26pt)
 -- (524.1pt, -288.26pt)
 -- (523.62pt, -288.26pt) -- cycle
;
\end{tikzpicture}
\begin{picture}(-5,0)(2.5,0)
\put(41.7,-293.2397){\fontsize{4.98}{1}\usefont{T1}{cmr}{m}{n}\selectfont\color{color_29791} }
\put(41.7,-308.4197){\fontsize{12.007}{1}\usefont{T2A}{cmr}{m}{n}\selectfont\color{color_29791}Если можно, то найти угол между первыми двумя элементами стандартного базиса. }
\put(41.7,-333.9197){\fontsize{12.0239}{1}\usefont{T1}{cmr}{m}{n}\selectfont\color{color_29791} Решение. Рассмотрим формулу (1) для пространства }
\put(366.12,-328.5197){\fontsize{10.0052}{1}\usefont{T1}{cmr}{m}{n}\selectfont\color{color_29791}2}
\put(356.88,-333.9197){\fontsize{11.9894}{1}\usefont{T1}{cmr}{m}{n}\selectfont\color{color_29791}. Эта формула ставит в соот-}
\put(41.71081,-359.5426){\fontsize{12.007}{1}\usefont{T2A}{cmr}{m}{n}\selectfont\color{color_29791}ветствие элементам }
\put(212.64,-354.1997){\fontsize{9.9771}{1}\usefont{T1}{cmr}{m}{n}\selectfont\color{color_29791}T}
\put(197.34,-359.5997){\fontsize{11.9808}{1}\usefont{T1}{cmr}{m}{n}\selectfont\color{color_29791}xxx)(}
\put(202.62,-363.7397){\fontsize{10.0052}{1}\usefont{T1}{cmr}{m}{n}\selectfont\color{color_29791}21}
\put(161.58,-359.5997){\fontsize{11.9937}{1}\usefont{T1}{cmr}{m}{n}\selectfont\color{color_29791}}
\put(221.58,-359.5397){\fontsize{12.007}{1}\usefont{T1}{cmr}{m}{n}\selectfont\color{color_29791}, }
\put(294.24,-354.1997){\fontsize{9.9771}{1}\usefont{T1}{cmr}{m}{n}\selectfont\color{color_29791}T}
\put(278.7,-359.5997){\fontsize{11.9808}{1}\usefont{T1}{cmr}{m}{n}\selectfont\color{color_29791}yyy)(}
\put(284.22,-363.7397){\fontsize{10.0052}{1}\usefont{T1}{cmr}{m}{n}\selectfont\color{color_29791}21}
\put(241.62,-359.5997){\fontsize{11.9937}{1}\usefont{T1}{cmr}{m}{n}\selectfont\color{color_29791}}
\put(302.94,-359.5397){\fontsize{12.007}{1}\usefont{T1}{cmr}{m}{n}\selectfont\color{color_29791} пространства }
\put(393.12,-354.1397){\fontsize{10.0052}{1}\usefont{T1}{cmr}{m}{n}\selectfont\color{color_29791}2}
\put(383.76,-359.5397){\fontsize{11.9894}{1}\usefont{T1}{cmr}{m}{n}\selectfont\color{color_29791} действительное число. }
\put(41.70714,-384.5023){\fontsize{12.007}{1}\usefont{T2A}{cmr}{m}{n}\selectfont\color{color_29791}Проверяем, удовлетворяет ли эта формула аксиомам 1–4 скалярного произведения. Сначала }
\put(41.70714,-405.3824){\fontsize{12.007}{1}\usefont{T2A}{cmr}{m}{n}\selectfont\color{color_29791}проверяем выполнение аксиомы 1. Поменяем местами множители }
\put(387.54,-405.3797){\fontsize{11.9808}{1}\usefont{T1}{cmr}{m}{n}\selectfont\color{color_29791}x и y: }
\put(41.69629,-429.2016){\fontsize{12.007}{1}\usefont{T1}{cmr}{m}{n}\selectfont\color{color_29791} }
\put(445.56,-433.3997){\fontsize{10.0052}{1}\usefont{T1}{cmr}{m}{n}\selectfont\color{color_29791}2212211122122111}
\put(421.92,-429.1997){\fontsize{12.007}{1}\usefont{T1}{cmr}{m}{n}\selectfont\color{color_29791}32243224),(yxyxyxyxxyxyxyxyxy. }
\put(41.70551,-454.4024){\fontsize{12.007}{1}\usefont{T2A}{cmr}{m}{n}\selectfont\color{color_29791}Получили выражение в правой части формулы }
\put(291.18,-454.3997){\fontsize{12.0504}{1}\usefont{T1}{cmr}{m}{n}\selectfont\color{color_29791}(1), т.е. ),(),(yxxy. Значит, аксиома 1 вы-}
\put(41.70669,-477.5012){\fontsize{12.007}{1}\usefont{T2A}{cmr}{m}{n}\selectfont\color{color_29791}полняется. Заметим, что выражение в правой части }
\put(309.48,-477.4997){\fontsize{12.0504}{1}\usefont{T1}{cmr}{m}{n}\selectfont\color{color_29791}(1) симметрическое относительно x и y. }
\put(41.69669,-500.6012){\fontsize{12.007}{1}\usefont{T2A}{cmr}{m}{n}\selectfont\color{color_29791}Оно не меняется при одновременной замене буквы }
\put(318.9,-500.5997){\fontsize{11.9808}{1}\usefont{T1}{cmr}{m}{n}\selectfont\color{color_29791}x на букву y, а буквы у на букву x. }
\put(41.69669,-523.5211){\fontsize{12.007}{1}\usefont{T2A}{cmr}{m}{n}\selectfont\color{color_29791}Это и обеспечивает коммутативность.  }
\put(41.68468,-544.2211){\fontsize{12.007}{1}\usefont{T1}{cmr}{m}{n}\selectfont\color{color_29791} Вместо аксиом 2 и 3 проверяем линейность по первому множителю. Для произвольных }
\put(91.92,-564.4396){\fontsize{10.0052}{1}\usefont{T1}{cmr}{m}{n}\selectfont\color{color_29791}2}
\put(60.78,-569.8397){\fontsize{12.007}{1}\usefont{T1}{cmr}{m}{n}\selectfont\color{color_29791},,zyx и любых чисел , получаем }
\put(41.70215,-593.7216){\fontsize{12.007}{1}\usefont{T1}{cmr}{m}{n}\selectfont\color{color_29791} }
\put(405.3,-593.7197){\fontsize{12.007}{1}\usefont{T1}{cmr}{m}{n}\selectfont\color{color_29791}}
\put(462.96,-597.9197){\fontsize{10.02}{1}\usefont{T1}{cmr}{m}{n}\selectfont\color{color_29791}222122211111}
\put(453.18,-593.7197){\fontsize{12.007}{1}\usefont{T1}{cmr}{m}{n}\selectfont\color{color_29791})(3)(2)(2)(4),(zyxzyxzyxzyxzyx }
\put(41.69605,-619.5828){\fontsize{12.007}{1}\usefont{T1}{cmr}{m}{n}\selectfont\color{color_29791} ),(),()3224()3224(}
\put(395.58,-623.7797){\fontsize{10.0052}{1}\usefont{T1}{cmr}{m}{n}\selectfont\color{color_29791}2212211122122111}
\put(483.54,-619.5797){\fontsize{12.0239}{1}\usefont{T1}{cmr}{m}{n}\selectfont\color{color_29791}zyzxzyzyzyzyzxzxzxzx. }
\put(41.69476,-644.6023){\fontsize{12.007}{1}\usefont{T2A}{cmr}{m}{n}\selectfont\color{color_29791}Линейность доказана, следовательно, аксиомы 2 и 3 выполняются. Вместо приведенного до-}
\put(41.69476,-665.3024){\fontsize{12.007}{1}\usefont{T2A}{cmr}{m}{n}\selectfont\color{color_29791}казательства достаточно заметить, что выражение в правой части }
\put(380.64,-665.2997){\fontsize{12.0504}{1}\usefont{T1}{cmr}{m}{n}\selectfont\color{color_29791}(1) линейно по переменным }
\put(48.3,-691.0997){\fontsize{10.0052}{1}\usefont{T1}{cmr}{m}{n}\selectfont\color{color_29791}1}
\put(44.1,-686.8997){\fontsize{12.0239}{1}\usefont{T1}{cmr}{m}{n}\selectfont\color{color_29791}x,}
\put(65.46,-691.0997){\fontsize{10.0052}{1}\usefont{T1}{cmr}{m}{n}\selectfont\color{color_29791}2}
\put(60.12,-686.8997){\fontsize{12.0239}{1}\usefont{T1}{cmr}{m}{n}\selectfont\color{color_29791}x:  }
\put(41.69391,-712.8228){\fontsize{12.007}{1}\usefont{T1}{cmr}{m}{n}\selectfont\color{color_29791} )32()24(3224}
\put(423.72,-717.0197){\fontsize{10.02}{1}\usefont{T1}{cmr}{m}{n}\selectfont\color{color_29791}21221122122111}
\put(418.26,-712.8197){\fontsize{12.0239}{1}\usefont{T1}{cmr}{m}{n}\selectfont\color{color_29791}yyxyyxyxyxyxyx. }
\put(41.69445,-737.7822){\fontsize{12.007}{1}\usefont{T2A}{cmr}{m}{n}\selectfont\color{color_29791}Проверяем выполнение аксиомы 4. Записываем скалярный квадрат и представляем получен-}
\put(41.69445,-758.4823){\fontsize{12.007}{1}\usefont{T2A}{cmr}{m}{n}\selectfont\color{color_29791}ную квадратичную форму в матричном виде (см. разд. 7 в [7]) }
\end{picture}
\newpage
\begin{tikzpicture}[overlay]\path(0pt,0pt);\end{tikzpicture}
\begin{picture}(-5,0)(2.5,0)
\put(41.7,-786.7997){\fontsize{12.007}{1}\usefont{T1}{cmr}{m}{n}\selectfont\color{color_29791} }
\put(276.665,-786.8597){\fontsize{12.007}{1}\usefont{T1}{cmr}{m}{n}\selectfont\color{color_29791}40}
\put(41.7,-68.66101){\fontsize{12.007}{1}\usefont{T1}{cmr}{m}{n}\selectfont\color{color_29791} }
\put(482.9399,-73.7597){\fontsize{11.9937}{1}\usefont{T1}{cmr}{m}{n}\selectfont\color{color_29791}}
\put(482.9399,-67.76282){\fontsize{11.9937}{1}\usefont{T1}{cmr}{m}{n}\selectfont\color{color_29791}}
\put(482.9399,-81.44769){\fontsize{11.9937}{1}\usefont{T1}{cmr}{m}{n}\selectfont\color{color_29791}}
\put(482.9399,-60.08691){\fontsize{11.9937}{1}\usefont{T1}{cmr}{m}{n}\selectfont\color{color_29791}}
\put(463.0184,-73.77173){\fontsize{11.9937}{1}\usefont{T1}{cmr}{m}{n}\selectfont\color{color_29791}}
\put(463.0184,-67.77478){\fontsize{11.9937}{1}\usefont{T1}{cmr}{m}{n}\selectfont\color{color_29791}}
\put(463.0184,-81.45972){\fontsize{11.9937}{1}\usefont{T1}{cmr}{m}{n}\selectfont\color{color_29791}}
\put(463.0184,-60.09888){\fontsize{11.9937}{1}\usefont{T1}{cmr}{m}{n}\selectfont\color{color_29791}}
\put(457.1415,-73.06409){\fontsize{11.9937}{1}\usefont{T1}{cmr}{m}{n}\selectfont\color{color_29791}}
\put(457.1415,-68.44647){\fontsize{11.9937}{1}\usefont{T1}{cmr}{m}{n}\selectfont\color{color_29791}}
\put(457.1415,-80.75201){\fontsize{11.9937}{1}\usefont{T1}{cmr}{m}{n}\selectfont\color{color_29791}}
\put(457.1415,-60.77051){\fontsize{11.9937}{1}\usefont{T1}{cmr}{m}{n}\selectfont\color{color_29791}}
\put(408.6629,-73.07611){\fontsize{11.9937}{1}\usefont{T1}{cmr}{m}{n}\selectfont\color{color_29791}}
\put(408.6629,-68.4585){\fontsize{11.9937}{1}\usefont{T1}{cmr}{m}{n}\selectfont\color{color_29791}}
\put(408.6629,-80.76398){\fontsize{11.9937}{1}\usefont{T1}{cmr}{m}{n}\selectfont\color{color_29791}}
\put(408.6629,-60.78247){\fontsize{11.9937}{1}\usefont{T1}{cmr}{m}{n}\selectfont\color{color_29791}}
\put(414.6598,-77.9455){\fontsize{11.9937}{1}\usefont{T1}{cmr}{m}{n}\selectfont\color{color_29791}}
\put(441.0579,-59.94299){\fontsize{11.9937}{1}\usefont{T1}{cmr}{m}{n}\selectfont\color{color_29791}}
\put(353.9357,-68.7583){\fontsize{11.9937}{1}\usefont{T1}{cmr}{m}{n}\selectfont\color{color_29791}}
\put(475.38,-81.3797){\fontsize{9.9947}{1}\usefont{T1}{cmr}{m}{n}\selectfont\color{color_29791}2}
\put(475.2001,-63.37921){\fontsize{9.9947}{1}\usefont{T1}{cmr}{m}{n}\selectfont\color{color_29791}1}
\put(396.6617,-72.8642){\fontsize{9.9947}{1}\usefont{T1}{cmr}{m}{n}\selectfont\color{color_29791}21}
\put(344.0996,-63.32928){\fontsize{9.9947}{1}\usefont{T1}{cmr}{m}{n}\selectfont\color{color_29791}2}
\put(343.44,-72.8642){\fontsize{9.9947}{1}\usefont{T1}{cmr}{m}{n}\selectfont\color{color_29791}221}
\put(272.2777,-63.32928){\fontsize{9.9947}{1}\usefont{T1}{cmr}{m}{n}\selectfont\color{color_29791}2}
\put(270.4787,-72.8642){\fontsize{9.9947}{1}\usefont{T1}{cmr}{m}{n}\selectfont\color{color_29791}1}
\put(239.3352,-63.32928){\fontsize{9.9947}{1}\usefont{T1}{cmr}{m}{n}\selectfont\color{color_29791}2}
\put(238.6156,-72.8642){\fontsize{9.9947}{1}\usefont{T1}{cmr}{m}{n}\selectfont\color{color_29791}21221}
\put(126.3551,-63.32928){\fontsize{9.9947}{1}\usefont{T1}{cmr}{m}{n}\selectfont\color{color_29791}2}
\put(124.616,-72.8642){\fontsize{9.9947}{1}\usefont{T1}{cmr}{m}{n}\selectfont\color{color_29791}1}
\put(445.62,-77.89972){\fontsize{11.9937}{1}\usefont{T1}{cmr}{m}{n}\selectfont\color{color_29791}32}
\put(450.1776,-59.89722){\fontsize{11.9937}{1}\usefont{T1}{cmr}{m}{n}\selectfont\color{color_29791}24}
\put(403.3182,-68.71252){\fontsize{11.9937}{1}\usefont{T1}{cmr}{m}{n}\selectfont\color{color_29791})(3443224),(}
\put(469.5594,-77.16809){\fontsize{11.9937}{1}\usefont{T1}{cmr}{m}{n}\selectfont\color{color_29791}x}
\put(470.5189,-59.16553){\fontsize{11.9937}{1}\usefont{T1}{cmr}{m}{n}\selectfont\color{color_29791}x}
\put(390.9047,-68.7005){\fontsize{11.9937}{1}\usefont{T1}{cmr}{m}{n}\selectfont\color{color_29791}xxxxxxxxxxxxxx}
\put(489.24,-68.65973){\fontsize{12.007}{1}\usefont{T1}{cmr}{m}{n}\selectfont\color{color_29791}. }
\put(41.70309,-103.516){\fontsize{12.007}{1}\usefont{T2A}{cmr}{m}{n}\selectfont\color{color_29791}Угловые миноры матрицы этой квадратичной формы положительные 04}
\put(477.4799,-107.7197){\fontsize{10.02}{1}\usefont{T1}{cmr}{m}{n}\selectfont\color{color_29791}1}
\put(503.58,-103.5197){\fontsize{12.007}{1}\usefont{T1}{cmr}{m}{n}\selectfont\color{color_29791}, }
\end{picture}
\begin{tikzpicture}[overlay]
\path(0pt,0pt);
\draw[color_29791,line width=0.998pt,line cap=round,line join=round]
(70.5pt, -119.42pt) -- (70.5pt, -151.46pt)
;
\draw[color_29791,line width=0.998pt,line cap=round,line join=round]
(117.78pt, -119.42pt) -- (117.78pt, -151.46pt)
;
\end{tikzpicture}
\begin{picture}(-5,0)(2.5,0)
\put(148.02,-138.4397){\fontsize{12.007}{1}\usefont{T1}{cmr}{m}{n}\selectfont\color{color_29791}08}
\put(104.3385,-147.613){\fontsize{12.007}{1}\usefont{T1}{cmr}{m}{n}\selectfont\color{color_29791}32}
\put(108.6611,-129.6146){\fontsize{12.007}{1}\usefont{T1}{cmr}{m}{n}\selectfont\color{color_29791}24}
\put(51.6,-142.6397){\fontsize{9.9947}{1}\usefont{T1}{cmr}{m}{n}\selectfont\color{color_29791}2}
\put(138.84,-138.4397){\fontsize{12.007}{1}\usefont{T1}{cmr}{m}{n}\selectfont\color{color_29791}}
\put(74.15829,-147.613){\fontsize{12.007}{1}\usefont{T1}{cmr}{m}{n}\selectfont\color{color_29791}}
\put(100.0214,-129.6146){\fontsize{12.007}{1}\usefont{T1}{cmr}{m}{n}\selectfont\color{color_29791}}
\put(60.48232,-138.4397){\fontsize{12.007}{1}\usefont{T1}{cmr}{m}{n}\selectfont\color{color_29791}. По критерию Сильвестра, квадратичная форма положительно опреде-}
\put(41.69136,-171.615){\fontsize{12.007}{1}\usefont{T2A}{cmr}{m}{n}\selectfont\color{color_29791}лена, т.е. 0),(}
\put(108,-171.6197){\fontsize{12.0239}{1}\usefont{T1}{cmr}{m}{n}\selectfont\color{color_29791}xx для всех ox. Значит, аксиома 4 для формулы (1) выполняется, посколь-}
\put(41.6931,-199.4639){\fontsize{12.007}{1}\usefont{T2A}{cmr}{m}{n}\selectfont\color{color_29791}ку }
\put(95.82,-199.3997){\fontsize{12.06}{1}\usefont{T1}{cmr}{m}{n}\selectfont\color{color_29791}0),(xx}
\put(103.86,-199.4597){\fontsize{12.007}{1}\usefont{T1}{cmr}{m}{n}\selectfont\color{color_29791} для всех }
\put(183.3,-194.0597){\fontsize{9.9947}{1}\usefont{T1}{cmr}{m}{n}\selectfont\color{color_29791}2}
\put(173.94,-199.4597){\fontsize{12.015}{1}\usefont{T1}{cmr}{m}{n}\selectfont\color{color_29791}x и }
\put(242.1,-199.3997){\fontsize{12.06}{1}\usefont{T1}{cmr}{m}{n}\selectfont\color{color_29791}0),(xx}
\put(249.96,-199.4597){\fontsize{12.007}{1}\usefont{T1}{cmr}{m}{n}\selectfont\color{color_29791} только при ox. Таким образом, формула (1) зада-}
\put(41.6999,-227.2439){\fontsize{12.007}{1}\usefont{T2A}{cmr}{m}{n}\selectfont\color{color_29791}ет скалярное произведение в }
\put(202.92,-221.8397){\fontsize{9.9947}{1}\usefont{T1}{cmr}{m}{n}\selectfont\color{color_29791}2}
\put(193.68,-227.2397){\fontsize{11.9894}{1}\usefont{T1}{cmr}{m}{n}\selectfont\color{color_29791}. }
\put(41.69795,-252.8626){\fontsize{12.007}{1}\usefont{T1}{cmr}{m}{n}\selectfont\color{color_29791} Находим угол }
\put(142.98,-252.7997){\fontsize{12.0738}{1}\usefont{T1}{cmr}{m}{n}\selectfont\color{color_29791}}
\put(152.34,-252.8597){\fontsize{12.007}{1}\usefont{T1}{cmr}{m}{n}\selectfont\color{color_29791} между первыми двумя векторами стандартного базиса }
\put(455.76,-247.4597){\fontsize{9.9947}{1}\usefont{T1}{cmr}{m}{n}\selectfont\color{color_29791}2}
\put(446.52,-252.8597){\fontsize{11.9894}{1}\usefont{T1}{cmr}{m}{n}\selectfont\color{color_29791}, т.е. между }
\put(41.70655,-280.764){\fontsize{12.007}{1}\usefont{T2A}{cmr}{m}{n}\selectfont\color{color_29791}векторами }
\put(150.54,-275.4197){\fontsize{9.9771}{1}\usefont{T1}{cmr}{m}{n}\selectfont\color{color_29791}T}
\put(99.72,-280.7596){\fontsize{11.9937}{1}\usefont{T1}{cmr}{m}{n}\selectfont\color{color_29791}e)01(}
\put(103.74,-284.9597){\fontsize{9.9947}{1}\usefont{T1}{cmr}{m}{n}\selectfont\color{color_29791}1}
\put(112.08,-280.7596){\fontsize{11.9937}{1}\usefont{T1}{cmr}{m}{n}\selectfont\color{color_29791} и }
\put(225.84,-275.4197){\fontsize{9.9771}{1}\usefont{T1}{cmr}{m}{n}\selectfont\color{color_29791}T}
\put(173.1,-280.7596){\fontsize{11.9937}{1}\usefont{T1}{cmr}{m}{n}\selectfont\color{color_29791}e)10(}
\put(178.2,-284.9597){\fontsize{9.9947}{1}\usefont{T1}{cmr}{m}{n}\selectfont\color{color_29791}2}
\put(187.32,-280.7596){\fontsize{11.9937}{1}\usefont{T1}{cmr}{m}{n}\selectfont\color{color_29791}. Вычисляем косинус угла по формуле }
\put(41.70762,-316.6365){\fontsize{12.007}{1}\usefont{T1}{cmr}{m}{n}\selectfont\color{color_29791} }
\end{picture}
\begin{tikzpicture}[overlay]
\path(0pt,0pt);
\draw[color_29791,line width=0.998pt,line cap=round,line join=round]
(104.76pt, -327.14pt) -- (106.08pt, -326.24pt)
;
\draw[color_29791,line width=1.997pt,line cap=round,line join=round]
(106.08pt, -326.78pt) -- (107.76pt, -331.58pt)
;
\draw[color_29791,line width=0.998pt,line cap=round,line join=round]
(108.3pt, -331.58pt) -- (111.48pt, -315.68pt)
 -- (111.48pt, -315.68pt)
 -- (143.46pt, -315.68pt)
;
\draw[color_29791,line width=0.998pt,line cap=round,line join=round]
(144.9pt, -327.14pt) -- (146.16pt, -326.24pt)
;
\draw[color_29791,line width=1.997pt,line cap=round,line join=round]
(146.16pt, -326.78pt) -- (147.9pt, -331.58pt)
;
\draw[color_29791,line width=0.998pt,line cap=round,line join=round]
(148.38pt, -331.58pt) -- (151.62pt, -315.68pt)
 -- (151.62pt, -315.68pt)
 -- (187.32pt, -315.68pt)
;
\draw[color_29791,line width=0.998pt,line cap=round,line join=round]
(103.26pt, -313.58pt) -- (188.34pt, -313.58pt)
;
\draw[color_29791,line width=0.998pt,line cap=round,line join=round]
(203.28pt, -327.02pt) -- (204.54pt, -326.18pt)
;
\draw[color_29791,line width=1.997pt,line cap=round,line join=round]
(204.54pt, -326.66pt) -- (206.28pt, -331.46pt)
;
\draw[color_29791,line width=0.998pt,line cap=round,line join=round]
(206.76pt, -331.46pt) -- (209.94pt, -315.68pt)
 -- (209.94pt, -315.68pt)
 -- (305.34pt, -315.68pt)
;
\draw[color_29791,line width=0.998pt,line cap=round,line join=round]
(306.78pt, -327.02pt) -- (308.04pt, -326.18pt)
;
\draw[color_29791,line width=1.997pt,line cap=round,line join=round]
(308.04pt, -326.66pt) -- (309.78pt, -331.46pt)
;
\draw[color_29791,line width=0.998pt,line cap=round,line join=round]
(310.26pt, -331.46pt) -- (313.44pt, -315.68pt)
 -- (313.44pt, -315.68pt)
 -- (408.84pt, -315.68pt)
;
\draw[color_29791,line width=0.998pt,line cap=round,line join=round]
(201.78pt, -313.58pt) -- (409.8pt, -313.58pt)
;
\draw[color_29791,line width=0.998pt,line cap=round,line join=round]
(430.92pt, -324.68pt) -- (432.24pt, -323.78pt)
;
\draw[color_29791,line width=1.997pt,line cap=round,line join=round]
(432.24pt, -324.32pt) -- (433.92pt, -327.62pt)
;
\draw[color_29791,line width=0.998pt,line cap=round,line join=round]
(434.46pt, -327.62pt) -- (437.64pt, -315.68pt)
 -- (437.64pt, -315.68pt)
 -- (444.06pt, -315.68pt)
;
\draw[color_29791,line width=0.998pt,line cap=round,line join=round]
(423.24pt, -313.58pt) -- (445.08pt, -313.58pt)
;
\draw[color_29791,line width=0.998pt,line cap=round,line join=round]
(468.66pt, -306.14pt) -- (469.92pt, -305.3pt)
;
\draw[color_29791,line width=1.997pt,line cap=round,line join=round]
(469.92pt, -305.78pt) -- (471.66pt, -309.08pt)
;
\draw[color_29791,line width=0.998pt,line cap=round,line join=round]
(472.14pt, -309.08pt) -- (475.32pt, -297.14pt)
 -- (475.32pt, -297.14pt)
 -- (481.8pt, -297.14pt)
;
\draw[color_29791,line width=0.998pt,line cap=round,line join=round]
(458.52pt, -313.58pt) -- (482.76pt, -313.58pt)
;
\end{tikzpicture}
\begin{picture}(-5,0)(2.5,0)
\put(467.82,-326.1797){\fontsize{12.007}{1}\usefont{T1}{cmr}{m}{n}\selectfont\color{color_29791}3}
\put(475.7446,-308.9016){\fontsize{12.007}{1}\usefont{T1}{cmr}{m}{n}\selectfont\color{color_29791}3}
\put(438.0066,-327.3804){\fontsize{12.007}{1}\usefont{T1}{cmr}{m}{n}\selectfont\color{color_29791}32}
\put(435.7253,-308.9016){\fontsize{12.007}{1}\usefont{T1}{cmr}{m}{n}\selectfont\color{color_29791}2}
\put(396.8466,-331.2827){\fontsize{12.007}{1}\usefont{T1}{cmr}{m}{n}\selectfont\color{color_29791}13102040301214}
\put(371.1036,-308.9016){\fontsize{12.007}{1}\usefont{T1}{cmr}{m}{n}\selectfont\color{color_29791}103002112014}
\put(182.7618,-327.3804){\fontsize{12.007}{1}\usefont{T1}{cmr}{m}{n}\selectfont\color{color_29791}),(),(}
\put(158.2195,-307.821){\fontsize{12.007}{1}\usefont{T1}{cmr}{m}{n}\selectfont\color{color_29791}),(}
\put(65.99374,-316.5861){\fontsize{12.007}{1}\usefont{T1}{cmr}{m}{n}\selectfont\color{color_29791}cos}
\put(402.24,-325.8797){\fontsize{10.02}{1}\usefont{T1}{cmr}{m}{n}\selectfont\color{color_29791}2222}
\put(176.6397,-331.5811){\fontsize{10.02}{1}\usefont{T1}{cmr}{m}{n}\selectfont\color{color_29791}2211}
\put(152.0406,-312.022){\fontsize{10.02}{1}\usefont{T1}{cmr}{m}{n}\selectfont\color{color_29791}21}
\put(459.3,-308.8997){\fontsize{12.007}{1}\usefont{T1}{cmr}{m}{n}\selectfont\color{color_29791}}
\put(448.6258,-316.5842){\fontsize{12.007}{1}\usefont{T1}{cmr}{m}{n}\selectfont\color{color_29791}}
\put(427.0252,-308.8997){\fontsize{12.007}{1}\usefont{T1}{cmr}{m}{n}\selectfont\color{color_29791}}
\put(413.3492,-316.5842){\fontsize{12.007}{1}\usefont{T1}{cmr}{m}{n}\selectfont\color{color_29791}}
\put(393.6097,-331.2807){\fontsize{12.007}{1}\usefont{T1}{cmr}{m}{n}\selectfont\color{color_29791}}
\put(367.8667,-308.8997){\fontsize{12.007}{1}\usefont{T1}{cmr}{m}{n}\selectfont\color{color_29791}}
\put(191.8921,-316.5842){\fontsize{12.007}{1}\usefont{T1}{cmr}{m}{n}\selectfont\color{color_29791}}
\put(171.54,-327.3797){\fontsize{12.0239}{1}\usefont{T1}{cmr}{m}{n}\selectfont\color{color_29791}eeee}
\put(146.9391,-307.8168){\fontsize{12.0239}{1}\usefont{T1}{cmr}{m}{n}\selectfont\color{color_29791}ee}
\put(485.34,-316.6396){\fontsize{12.007}{1}\usefont{T1}{cmr}{m}{n}\selectfont\color{color_29791}.  }
\put(41.70536,-363.563){\fontsize{12.007}{1}\usefont{T2A}{cmr}{m}{n}\selectfont\color{color_29791}Значит, угол между векторами }
\end{picture}
\begin{tikzpicture}[overlay]
\path(0pt,0pt);
\draw[color_29791,line width=0.998pt,line cap=round,line join=round]
(273.78pt, -353.12pt) -- (275.04pt, -352.22pt)
;
\draw[color_29791,line width=1.997pt,line cap=round,line join=round]
(275.04pt, -352.76pt) -- (276.78pt, -356pt)
;
\draw[color_29791,line width=0.998pt,line cap=round,line join=round]
(277.26pt, -356pt) -- (280.44pt, -344.12pt)
 -- (280.44pt, -344.12pt)
 -- (286.92pt, -344.12pt)
;
\draw[color_29791,line width=0.998pt,line cap=round,line join=round]
(272.28pt, -360.56pt) -- (287.88pt, -360.56pt)
;
\end{tikzpicture}
\begin{picture}(-5,0)(2.5,0)
\put(277.26,-373.0997){\fontsize{12.007}{1}\usefont{T1}{cmr}{m}{n}\selectfont\color{color_29791}3}
\put(280.8621,-355.8216){\fontsize{12.007}{1}\usefont{T1}{cmr}{m}{n}\selectfont\color{color_29791}3}
\put(240.1824,-363.5661){\fontsize{12.007}{1}\usefont{T1}{cmr}{m}{n}\selectfont\color{color_29791}arccos. }
\put(41.69467,-398.3624){\fontsize{12.007}{1}\usefont{T1}{cmr}{m}{n}\selectfont\color{color_29791} Рассмотрим формулу }
\put(193.5,-398.3597){\fontsize{12.0504}{1}\usefont{T1}{cmr}{m}{n}\selectfont\color{color_29791}(2) для пространства }
\put(316.68,-392.9597){\fontsize{9.9947}{1}\usefont{T1}{cmr}{m}{n}\selectfont\color{color_29791}2}
\put(307.44,-398.3597){\fontsize{11.9894}{1}\usefont{T1}{cmr}{m}{n}\selectfont\color{color_29791}. Выражение в правой части формулы }
\put(41.7,-419.8997){\fontsize{12.0504}{1}\usefont{T1}{cmr}{m}{n}\selectfont\color{color_29791}(2) симметрическое относительно x и y, а также линейно по переменным }
\put(436.26,-424.0997){\fontsize{9.9947}{1}\usefont{T1}{cmr}{m}{n}\selectfont\color{color_29791}1}
\put(432.06,-419.8997){\fontsize{12.0239}{1}\usefont{T1}{cmr}{m}{n}\selectfont\color{color_29791}x,}
\put(453.42,-424.0997){\fontsize{9.9947}{1}\usefont{T1}{cmr}{m}{n}\selectfont\color{color_29791}2}
\put(448.08,-419.8997){\fontsize{12.0239}{1}\usefont{T1}{cmr}{m}{n}\selectfont\color{color_29791}x. Значит, ак-}
\put(41.69574,-444.9223){\fontsize{12.007}{1}\usefont{T2A}{cmr}{m}{n}\selectfont\color{color_29791}сиомы 1–3 выполняются. Проверяем выполнение аксиомы 4. Записываем скалярный квадрат }
\put(41.68373,-465.6224){\fontsize{12.007}{1}\usefont{T2A}{cmr}{m}{n}\selectfont\color{color_29791}и представляем полученную квадратичную форму в матричном виде  }
\put(41.68373,-497.2488){\fontsize{12.007}{1}\usefont{T1}{cmr}{m}{n}\selectfont\color{color_29791} }
\put(402.6,-502.3397){\fontsize{11.9937}{1}\usefont{T1}{cmr}{m}{n}\selectfont\color{color_29791}}
\put(402.6,-496.3428){\fontsize{11.9937}{1}\usefont{T1}{cmr}{m}{n}\selectfont\color{color_29791}}
\put(402.6,-510.0277){\fontsize{11.9937}{1}\usefont{T1}{cmr}{m}{n}\selectfont\color{color_29791}}
\put(402.6,-488.6669){\fontsize{11.9937}{1}\usefont{T1}{cmr}{m}{n}\selectfont\color{color_29791}}
\put(383.638,-502.3517){\fontsize{11.9937}{1}\usefont{T1}{cmr}{m}{n}\selectfont\color{color_29791}}
\put(383.638,-496.3548){\fontsize{11.9937}{1}\usefont{T1}{cmr}{m}{n}\selectfont\color{color_29791}}
\put(383.638,-510.0397){\fontsize{11.9937}{1}\usefont{T1}{cmr}{m}{n}\selectfont\color{color_29791}}
\put(383.638,-488.6789){\fontsize{11.9937}{1}\usefont{T1}{cmr}{m}{n}\selectfont\color{color_29791}}
\put(379.1403,-501.704){\fontsize{11.9937}{1}\usefont{T1}{cmr}{m}{n}\selectfont\color{color_29791}}
\put(379.1403,-497.0865){\fontsize{11.9937}{1}\usefont{T1}{cmr}{m}{n}\selectfont\color{color_29791}}
\put(379.1403,-509.392){\fontsize{11.9937}{1}\usefont{T1}{cmr}{m}{n}\selectfont\color{color_29791}}
\put(379.1403,-489.4105){\fontsize{11.9937}{1}\usefont{T1}{cmr}{m}{n}\selectfont\color{color_29791}}
\put(331.6213,-501.716){\fontsize{11.9937}{1}\usefont{T1}{cmr}{m}{n}\selectfont\color{color_29791}}
\put(331.6213,-497.0985){\fontsize{11.9937}{1}\usefont{T1}{cmr}{m}{n}\selectfont\color{color_29791}}
\put(331.6213,-509.404){\fontsize{11.9937}{1}\usefont{T1}{cmr}{m}{n}\selectfont\color{color_29791}}
\put(331.6213,-489.4225){\fontsize{11.9937}{1}\usefont{T1}{cmr}{m}{n}\selectfont\color{color_29791}}
\put(363.8963,-506.5255){\fontsize{11.9937}{1}\usefont{T1}{cmr}{m}{n}\selectfont\color{color_29791}}
\put(363.5365,-488.523){\fontsize{11.9937}{1}\usefont{T1}{cmr}{m}{n}\selectfont\color{color_29791}}
\put(280.4921,-497.3383){\fontsize{11.9937}{1}\usefont{T1}{cmr}{m}{n}\selectfont\color{color_29791}}
\put(395.46,-509.9597){\fontsize{9.9947}{1}\usefont{T1}{cmr}{m}{n}\selectfont\color{color_29791}2}
\put(395.3401,-492.0192){\fontsize{9.9947}{1}\usefont{T1}{cmr}{m}{n}\selectfont\color{color_29791}1}
\put(321.5392,-501.5042){\fontsize{9.9947}{1}\usefont{T1}{cmr}{m}{n}\selectfont\color{color_29791}21}
\put(271.3158,-491.9692){\fontsize{9.9947}{1}\usefont{T1}{cmr}{m}{n}\selectfont\color{color_29791}2}
\put(270.6562,-501.5042){\fontsize{9.9947}{1}\usefont{T1}{cmr}{m}{n}\selectfont\color{color_29791}221}
\put(205.2609,-491.9692){\fontsize{9.9947}{1}\usefont{T1}{cmr}{m}{n}\selectfont\color{color_29791}2}
\put(203.5218,-501.5042){\fontsize{9.9947}{1}\usefont{T1}{cmr}{m}{n}\selectfont\color{color_29791}1}
\put(372.18,-506.4796){\fontsize{11.9937}{1}\usefont{T1}{cmr}{m}{n}\selectfont\color{color_29791}34}
\put(372.18,-488.4771){\fontsize{11.9937}{1}\usefont{T1}{cmr}{m}{n}\selectfont\color{color_29791}41}
\put(327.7194,-497.2924){\fontsize{11.9937}{1}\usefont{T1}{cmr}{m}{n}\selectfont\color{color_29791})(382),(}
\put(390.1825,-505.748){\fontsize{11.9937}{1}\usefont{T1}{cmr}{m}{n}\selectfont\color{color_29791}x}
\put(391.142,-487.8054){\fontsize{11.9937}{1}\usefont{T1}{cmr}{m}{n}\selectfont\color{color_29791}x}
\put(316.2654,-497.2804){\fontsize{11.9937}{1}\usefont{T1}{cmr}{m}{n}\selectfont\color{color_29791}xxxxxxxx}
\put(409.26,-497.2397){\fontsize{12.007}{1}\usefont{T1}{cmr}{m}{n}\selectfont\color{color_29791}.  }
\put(41.70172,-532.096){\fontsize{12.007}{1}\usefont{T2A}{cmr}{m}{n}\selectfont\color{color_29791}Вычисляем угловые миноры матрицы квадратичной формы 01}
\put(361.8,-536.2997){\fontsize{9.9947}{1}\usefont{T1}{cmr}{m}{n}\selectfont\color{color_29791}1}
\put(386.58,-532.0997){\fontsize{12.007}{1}\usefont{T1}{cmr}{m}{n}\selectfont\color{color_29791}, 019}
\put(420,-536.2997){\fontsize{9.9947}{1}\usefont{T1}{cmr}{m}{n}\selectfont\color{color_29791}2}
\put(460.26,-532.0997){\fontsize{12.007}{1}\usefont{T1}{cmr}{m}{n}\selectfont\color{color_29791}. По кри-}
\put(41.69598,-557.1223){\fontsize{12.007}{1}\usefont{T2A}{cmr}{m}{n}\selectfont\color{color_29791}терию Сильвестра, эта квадратичная форма не является положительно определенной. Значит, }
\put(41.67197,-577.8224){\fontsize{12.007}{1}\usefont{T2A}{cmr}{m}{n}\selectfont\color{color_29791}аксиома 4 не выполняется. Чтобы в этом убедиться, не обязательно использовать критерий }
\put(41.67197,-598.7025){\fontsize{12.007}{1}\usefont{T2A}{cmr}{m}{n}\selectfont\color{color_29791}Сильвестра. Достаточно привести пример ненулевого вектора }
\put(368.46,-598.6997){\fontsize{11.9937}{1}\usefont{T1}{cmr}{m}{n}\selectfont\color{color_29791}x, для которого 0),(xx. На-}
\put(41.69731,-626.5439){\fontsize{12.007}{1}\usefont{T2A}{cmr}{m}{n}\selectfont\color{color_29791}пример, для }
\put(156.72,-621.1397){\fontsize{9.9771}{1}\usefont{T1}{cmr}{m}{n}\selectfont\color{color_29791}T}
\put(110.16,-626.5397){\fontsize{12.0239}{1}\usefont{T1}{cmr}{m}{n}\selectfont\color{color_29791}x)10( имеем 3),(xx. Значит, аксиома 4 не выполняется. Поэтому фор-}
\put(41.70548,-654.3239){\fontsize{12.007}{1}\usefont{T2A}{cmr}{m}{n}\selectfont\color{color_29791}мулой }
\put(76.74,-654.3197){\fontsize{12.0504}{1}\usefont{T1}{cmr}{m}{n}\selectfont\color{color_29791}(2) нельзя задать скалярное произведение в }
\put(312.96,-648.9197){\fontsize{9.9947}{1}\usefont{T1}{cmr}{m}{n}\selectfont\color{color_29791}2}
\put(303.72,-654.3197){\fontsize{11.9894}{1}\usefont{T1}{cmr}{m}{n}\selectfont\color{color_29791}.  }
\put(41.7058,-675.9203){\fontsize{12.007}{1}\usefont{T1}{cmr}{m}{n}\selectfont\color{color_29791} Рассмотрим формулу }
\put(195.6,-675.9196){\fontsize{12.0504}{1}\usefont{T1}{cmr}{m}{n}\selectfont\color{color_29791}(1) для пространства }
\put(318.54,-680.1197){\fontsize{10.02}{1}\usefont{T1}{cmr}{m}{n}\selectfont\color{color_29791}2}
\put(313.14,-675.9196){\fontsize{12.0239}{1}\usefont{T1}{cmr}{m}{n}\selectfont\color{color_29791}P. Эта формула ставит в соответствие }
\put(41.71427,-705.8651){\fontsize{12.007}{1}\usefont{T2A}{cmr}{m}{n}\selectfont\color{color_29791}элементам }
\put(186.24,-705.8596){\fontsize{12.0239}{1}\usefont{T1}{cmr}{m}{n}\selectfont\color{color_29791}cbxaxxp}
\put(147.54,-700.4597){\fontsize{9.9947}{1}\usefont{T1}{cmr}{m}{n}\selectfont\color{color_29791}2}
\put(118.8,-705.8596){\fontsize{12.007}{1}\usefont{T1}{cmr}{m}{n}\selectfont\color{color_29791})(, xxxq}
\put(248.58,-700.4597){\fontsize{9.9947}{1}\usefont{T1}{cmr}{m}{n}\selectfont\color{color_29791}2}
\put(218.28,-705.8596){\fontsize{12.007}{1}\usefont{T1}{cmr}{m}{n}\selectfont\color{color_29791})( пространства }
\put(377.52,-710.1197){\fontsize{10.02}{1}\usefont{T1}{cmr}{m}{n}\selectfont\color{color_29791}2}
\put(372.12,-705.9196){\fontsize{12.0239}{1}\usefont{T1}{cmr}{m}{n}\selectfont\color{color_29791}P}
\put(384.96,-705.8596){\fontsize{12.007}{1}\usefont{T1}{cmr}{m}{n}\selectfont\color{color_29791} действительное число.  }
\put(41.69188,-730.8222){\fontsize{12.007}{1}\usefont{T2A}{cmr}{m}{n}\selectfont\color{color_29791}Проверяем, удовлетворяет ли эта формула аксиомам 1–4 скалярного произведения. Сначала }
\put(41.69188,-751.5222){\fontsize{12.007}{1}\usefont{T2A}{cmr}{m}{n}\selectfont\color{color_29791}проверяем выполнение аксиомы 1. Выражение в правой части }
\put(377.22,-751.5197){\fontsize{12.0504}{1}\usefont{T1}{cmr}{m}{n}\selectfont\color{color_29791}(1) симметрическое относи-}
\end{picture}
\newpage
\begin{tikzpicture}[overlay]\path(0pt,0pt);\end{tikzpicture}
\begin{picture}(-5,0)(2.5,0)
\put(41.7,-786.7997){\fontsize{12.007}{1}\usefont{T1}{cmr}{m}{n}\selectfont\color{color_29791} }
\put(276.665,-786.8597){\fontsize{12.007}{1}\usefont{T1}{cmr}{m}{n}\selectfont\color{color_29791}41}
\put(41.7,-57.91479){\fontsize{12.007}{1}\usefont{T2A}{cmr}{m}{n}\selectfont\color{color_29791}тельно }
\put(82.68,-57.91968){\fontsize{12.0239}{1}\usefont{T1}{cmr}{m}{n}\selectfont\color{color_29791}p и q. Действительно, при одновременной замене буквы p на букву q, а буквы q – }
\put(41.7003,-81.02118){\fontsize{12.007}{1}\usefont{T2A}{cmr}{m}{n}\selectfont\color{color_29791}на букву }
\put(92.34,-81.01971){\fontsize{12.0239}{1}\usefont{T1}{cmr}{m}{n}\selectfont\color{color_29791}p, выражение не меняется }
\put(41.70141,-104.1212){\fontsize{12.007}{1}\usefont{T1}{cmr}{m}{n}\selectfont\color{color_29791} }
\put(435.12,-104.2997){\fontsize{11.565}{1}\usefont{T1}{cmr}{m}{n}\selectfont\color{color_29791})0()0(2)1()1()1()1()0()0(2)1()1()1()1(qpqpqppqpqpq}
\put(422.7108,-103.6405){\fontsize{11.565}{1}\usefont{T1}{cmr}{m}{n}\selectfont\color{color_29791}}
\put(400.26,-103.6397){\fontsize{11.565}{1}\usefont{T1}{cmr}{m}{n}\selectfont\color{color_29791}}
\put(378.66,-104.2997){\fontsize{11.565}{1}\usefont{T1}{cmr}{m}{n}\selectfont\color{color_29791}}
\put(344.46,-103.6397){\fontsize{11.565}{1}\usefont{T1}{cmr}{m}{n}\selectfont\color{color_29791}}
\put(328.62,-104.2997){\fontsize{11.565}{1}\usefont{T1}{cmr}{m}{n}\selectfont\color{color_29791}}
\put(312.42,-103.6397){\fontsize{11.565}{1}\usefont{T1}{cmr}{m}{n}\selectfont\color{color_29791}}
\put(277.74,-104.2997){\fontsize{11.565}{1}\usefont{T1}{cmr}{m}{n}\selectfont\color{color_29791}}
\put(258.54,-103.6397){\fontsize{11.565}{1}\usefont{T1}{cmr}{m}{n}\selectfont\color{color_29791}}
\put(214.5,-104.2997){\fontsize{11.565}{1}\usefont{T1}{cmr}{m}{n}\selectfont\color{color_29791}}
\put(178.8567,-103.6405){\fontsize{11.565}{1}\usefont{T1}{cmr}{m}{n}\selectfont\color{color_29791}}
\put(164.46,-104.2997){\fontsize{11.565}{1}\usefont{T1}{cmr}{m}{n}\selectfont\color{color_29791}}
\put(148.26,-103.6397){\fontsize{11.565}{1}\usefont{T1}{cmr}{m}{n}\selectfont\color{color_29791}}
\put(441.24,-104.1197){\fontsize{12.007}{1}\usefont{T1}{cmr}{m}{n}\selectfont\color{color_29791}. }
\put(41.69507,-127.0411){\fontsize{12.007}{1}\usefont{T2A}{cmr}{m}{n}\selectfont\color{color_29791}Значит, формула }
\put(130.92,-127.0397){\fontsize{12.0504}{1}\usefont{T1}{cmr}{m}{n}\selectfont\color{color_29791}(1) удовлетворяет аксиоме 1.  }
\put(41.69984,-147.9199){\fontsize{12.007}{1}\usefont{T1}{cmr}{m}{n}\selectfont\color{color_29791} Выражение в правой части }
\put(210.36,-147.9197){\fontsize{12.0504}{1}\usefont{T1}{cmr}{m}{n}\selectfont\color{color_29791}(1) линейно по многочлену p, т.е. формула (1) линейна по }
\put(41.70792,-170.8411){\fontsize{12.007}{1}\usefont{T2A}{cmr}{m}{n}\selectfont\color{color_29791}первому множителю (аксиомы 2–3 выполняются). Проверяем выполнение аксиомы 4. Запи-}
\put(41.70792,-195.5034){\fontsize{12.007}{1}\usefont{T2A}{cmr}{m}{n}\selectfont\color{color_29791}сываем скалярный квадрат }
\put(324.6,-190.0997){\fontsize{9.99}{1}\usefont{T1}{cmr}{m}{n}\selectfont\color{color_29791}2}
\put(316.8,-195.4997){\fontsize{11.9878}{1}\usefont{T1}{cmr}{m}{n}\selectfont\color{color_29791})]0([2)1()1(2),(ppppp}
\put(303.8891,-194.9003){\fontsize{11.9878}{1}\usefont{T1}{cmr}{m}{n}\selectfont\color{color_29791}}
\put(277.6119,-195.4997){\fontsize{11.9878}{1}\usefont{T1}{cmr}{m}{n}\selectfont\color{color_29791}}
\put(260.2775,-194.9003){\fontsize{11.9878}{1}\usefont{T1}{cmr}{m}{n}\selectfont\color{color_29791}}
\put(215.9347,-195.4997){\fontsize{11.9878}{1}\usefont{T1}{cmr}{m}{n}\selectfont\color{color_29791}. Это выражение может быть отрица-}
\put(41.6988,-218.6012){\fontsize{12.007}{1}\usefont{T2A}{cmr}{m}{n}\selectfont\color{color_29791}тельным. Например, для многочлена }
\put(294.06,-218.5997){\fontsize{12.06}{1}\usefont{T1}{cmr}{m}{n}\selectfont\color{color_29791}1)(xxp имеем 2),(pp. Значит, аксиома 4 не }
\put(41.70641,-242.4216){\fontsize{12.007}{1}\usefont{T2A}{cmr}{m}{n}\selectfont\color{color_29791}выполняется. Поэтому формулой }
\put(215.7,-242.4197){\fontsize{12.0504}{1}\usefont{T1}{cmr}{m}{n}\selectfont\color{color_29791}(1) нельзя задать скалярное произведение в }
\put(448.5,-246.6197){\fontsize{10.02}{1}\usefont{T1}{cmr}{m}{n}\selectfont\color{color_29791}2}
\put(443.1,-242.4197){\fontsize{12.0239}{1}\usefont{T1}{cmr}{m}{n}\selectfont\color{color_29791}P. }
\put(41.6985,-268.3428){\fontsize{12.007}{1}\usefont{T1}{cmr}{m}{n}\selectfont\color{color_29791} Рассмотрим формулу }
\put(195,-268.3397){\fontsize{12.0504}{1}\usefont{T1}{cmr}{m}{n}\selectfont\color{color_29791}(2) для пространства }
\put(317.04,-272.5397){\fontsize{9.99}{1}\usefont{T1}{cmr}{m}{n}\selectfont\color{color_29791}2}
\put(311.64,-268.3397){\fontsize{12.0239}{1}\usefont{T1}{cmr}{m}{n}\selectfont\color{color_29791}P. Правая часть формулы симметриче-}
\put(41.71515,-293.5424){\fontsize{12.007}{1}\usefont{T2A}{cmr}{m}{n}\selectfont\color{color_29791}ская относительно }
\put(143.94,-293.5397){\fontsize{12.0239}{1}\usefont{T1}{cmr}{m}{n}\selectfont\color{color_29791}p и q, а также линейна по p (из-за линейности интеграла). Поэтому ак-}
\put(41.71802,-316.4611){\fontsize{12.007}{1}\usefont{T2A}{cmr}{m}{n}\selectfont\color{color_29791}сиомы 1–3 выполняются. Проверяем выполнение аксиомы 4. Записываем скалярный квадрат }
\put(41.71802,-352.0378){\fontsize{12.007}{1}\usefont{T1}{cmr}{m}{n}\selectfont\color{color_29791} }
\put(378.72,-346.6997){\fontsize{9.99}{1}\usefont{T1}{cmr}{m}{n}\selectfont\color{color_29791}2}
\put(220.7981,-336.6198){\fontsize{9.99}{1}\usefont{T1}{cmr}{m}{n}\selectfont\color{color_29791}1}
\put(223.855,-369.0773){\fontsize{9.99}{1}\usefont{T1}{cmr}{m}{n}\selectfont\color{color_29791}1}
\put(312.8959,-346.6997){\fontsize{9.99}{1}\usefont{T1}{cmr}{m}{n}\selectfont\color{color_29791}22}
\put(366.9,-352.0397){\fontsize{12.007}{1}\usefont{T1}{cmr}{m}{n}\selectfont\color{color_29791}))]1([\})]([)](\{[),(pdxxpxppp}
\put(291.6566,-351.3193){\fontsize{12.007}{1}\usefont{T1}{cmr}{m}{n}\selectfont\color{color_29791}}
\put(271.617,-352.0397){\fontsize{12.007}{1}\usefont{T1}{cmr}{m}{n}\selectfont\color{color_29791}}
\put(220.68,-357.1997){\fontsize{19.9999}{1}\usefont{T1}{cmr}{m}{n}\selectfont\color{color_29791}}
\put(218.88,-369.0797){\fontsize{9.9947}{1}\usefont{T1}{cmr}{m}{n}\selectfont\color{color_29791}}
\put(385.74,-352.0397){\fontsize{12.007}{1}\usefont{T1}{cmr}{m}{n}\selectfont\color{color_29791}. }
\put(41.70343,-389.0573){\fontsize{12.007}{1}\usefont{T2A}{cmr}{m}{n}\selectfont\color{color_29791}Определенный интеграл от неотрицательной функции имеет неотрицательное значение. По-}
\put(41.72744,-409.9375){\fontsize{12.007}{1}\usefont{T2A}{cmr}{m}{n}\selectfont\color{color_29791}этому 0),(}
\put(94.02,-409.9397){\fontsize{12.0239}{1}\usefont{T1}{cmr}{m}{n}\selectfont\color{color_29791}pp. Предположим, что 0),(pp, тогда }
\put(41.69323,-447.7377){\fontsize{12.007}{1}\usefont{T1}{cmr}{m}{n}\selectfont\color{color_29791} 0))]1([)]([)]([),(}
\put(374.7,-442.3997){\fontsize{9.99}{1}\usefont{T1}{cmr}{m}{n}\selectfont\color{color_29791}2}
\put(274.4404,-432.2598){\fontsize{9.99}{1}\usefont{T1}{cmr}{m}{n}\selectfont\color{color_29791}1}
\put(277.4973,-464.7773){\fontsize{9.99}{1}\usefont{T1}{cmr}{m}{n}\selectfont\color{color_29791}1}
\put(313.6811,-442.3997){\fontsize{9.99}{1}\usefont{T1}{cmr}{m}{n}\selectfont\color{color_29791}2}
\put(205.859,-432.2598){\fontsize{9.99}{1}\usefont{T1}{cmr}{m}{n}\selectfont\color{color_29791}1}
\put(208.916,-464.7773){\fontsize{9.99}{1}\usefont{T1}{cmr}{m}{n}\selectfont\color{color_29791}1}
\put(242.5723,-442.3997){\fontsize{9.99}{1}\usefont{T1}{cmr}{m}{n}\selectfont\color{color_29791}2}
\put(383.82,-447.7397){\fontsize{12.007}{1}\usefont{T1}{cmr}{m}{n}\selectfont\color{color_29791}}
\put(292.5668,-447.0793){\fontsize{12.007}{1}\usefont{T1}{cmr}{m}{n}\selectfont\color{color_29791}}
\put(263.6419,-447.7397){\fontsize{12.007}{1}\usefont{T1}{cmr}{m}{n}\selectfont\color{color_29791}}
\put(274.26,-452.8997){\fontsize{19.9999}{1}\usefont{T1}{cmr}{m}{n}\selectfont\color{color_29791}}
\put(272.52,-464.7797){\fontsize{9.9947}{1}\usefont{T1}{cmr}{m}{n}\selectfont\color{color_29791}}
\put(348.48,-447.7397){\fontsize{12.0239}{1}\usefont{T1}{cmr}{m}{n}\selectfont\color{color_29791}pdxxpdxxppp. }
\put(41.70641,-484.7573){\fontsize{12.007}{1}\usefont{T2A}{cmr}{m}{n}\selectfont\color{color_29791}Следовательно, каждое слагаемое равно нулю. Так как многочлен является непрерывной }
\put(41.74243,-520.334){\fontsize{12.007}{1}\usefont{T2A}{cmr}{m}{n}\selectfont\color{color_29791}функцией, равенство 0)]([}
\put(161.04,-504.8596){\fontsize{9.99}{1}\usefont{T1}{cmr}{m}{n}\selectfont\color{color_29791}1}
\put(164.0969,-537.377){\fontsize{9.99}{1}\usefont{T1}{cmr}{m}{n}\selectfont\color{color_29791}1}
\put(197.6933,-514.9995){\fontsize{9.99}{1}\usefont{T1}{cmr}{m}{n}\selectfont\color{color_29791}2}
\put(219.84,-520.3397){\fontsize{12.007}{1}\usefont{T1}{cmr}{m}{n}\selectfont\color{color_29791}}
\put(160.86,-525.4997){\fontsize{19.9999}{1}\usefont{T1}{cmr}{m}{n}\selectfont\color{color_29791}}
\put(159.12,-537.3796){\fontsize{9.9947}{1}\usefont{T1}{cmr}{m}{n}\selectfont\color{color_29791}}
\put(205.26,-520.3397){\fontsize{12.0239}{1}\usefont{T1}{cmr}{m}{n}\selectfont\color{color_29791}dxxp возможно только для нулевого многочлена 0)(xp }
\put(41.69504,-557.3573){\fontsize{12.007}{1}\usefont{T1}{cmr}{m}{n}\selectfont\color{color_29791}(см. пример 3 евклидовых пространств). Следовательно, аксиома 4 выполняется. Таким обра-}
\put(41.68303,-578.9579){\fontsize{12.007}{1}\usefont{T2A}{cmr}{m}{n}\selectfont\color{color_29791}зом, формула }
\put(113.76,-578.9597){\fontsize{12.0504}{1}\usefont{T1}{cmr}{m}{n}\selectfont\color{color_29791}(2) задает скалярное произведение в }
\put(309.96,-583.2197){\fontsize{10.02}{1}\usefont{T1}{cmr}{m}{n}\selectfont\color{color_29791}2}
\put(304.56,-579.0197){\fontsize{12.0239}{1}\usefont{T1}{cmr}{m}{n}\selectfont\color{color_29791}P}
\put(317.4,-578.9597){\fontsize{12.007}{1}\usefont{T1}{cmr}{m}{n}\selectfont\color{color_29791}. }
\put(41.69527,-604.8228){\fontsize{12.007}{1}\usefont{T1}{cmr}{m}{n}\selectfont\color{color_29791} Находим угол  между первыми двумя элементами стандартного базиса }
\put(456,-609.0197){\fontsize{10.02}{1}\usefont{T1}{cmr}{m}{n}\selectfont\color{color_29791}2}
\put(450.6,-604.8197){\fontsize{12.0239}{1}\usefont{T1}{cmr}{m}{n}\selectfont\color{color_29791}P, т.е. между }
\put(41.70613,-630.7428){\fontsize{12.007}{1}\usefont{T2A}{cmr}{m}{n}\selectfont\color{color_29791}многочленами 1)(}
\put(127.02,-634.9397){\fontsize{9.99}{1}\usefont{T1}{cmr}{m}{n}\selectfont\color{color_29791}1}
\put(150,-630.7397){\fontsize{12.007}{1}\usefont{T1}{cmr}{m}{n}\selectfont\color{color_29791}xp и xxp)(}
\put(187.44,-634.9397){\fontsize{9.99}{1}\usefont{T1}{cmr}{m}{n}\selectfont\color{color_29791}2}
\put(228.18,-630.7397){\fontsize{12.007}{1}\usefont{T1}{cmr}{m}{n}\selectfont\color{color_29791}. Вычисляем скалярные произведения }
\put(41.71129,-670.699){\fontsize{12.007}{1}\usefont{T1}{cmr}{m}{n}\selectfont\color{color_29791} }
\put(286.68,-670.6996){\fontsize{12.0209}{1}\usefont{T1}{cmr}{m}{n}\selectfont\color{color_29791}1111]101[),(}
\put(235.08,-655.0997){\fontsize{10.0172}{1}\usefont{T1}{cmr}{m}{n}\selectfont\color{color_29791}1}
\put(238.2054,-687.8559){\fontsize{10.0172}{1}\usefont{T1}{cmr}{m}{n}\selectfont\color{color_29791}1}
\put(125.9426,-655.0997){\fontsize{10.0172}{1}\usefont{T1}{cmr}{m}{n}\selectfont\color{color_29791}1}
\put(129.068,-687.8559){\fontsize{10.0172}{1}\usefont{T1}{cmr}{m}{n}\selectfont\color{color_29791}1}
\put(100.569,-674.8937){\fontsize{10.0172}{1}\usefont{T1}{cmr}{m}{n}\selectfont\color{color_29791}21}
\put(277.98,-670.6996){\fontsize{12.0209}{1}\usefont{T1}{cmr}{m}{n}\selectfont\color{color_29791}}
\put(234.9,-675.8596){\fontsize{20.0346}{1}\usefont{T1}{cmr}{m}{n}\selectfont\color{color_29791}}
\put(233.04,-687.8596){\fontsize{10.0172}{1}\usefont{T1}{cmr}{m}{n}\selectfont\color{color_29791}}
\put(242.88,-670.6996){\fontsize{12.0209}{1}\usefont{T1}{cmr}{m}{n}\selectfont\color{color_29791}xdxdxxpp;  3121]01[),(}
\put(431.7,-665.2997){\fontsize{10.0172}{1}\usefont{T1}{cmr}{m}{n}\selectfont\color{color_29791}2}
\put(356.6411,-655.1022){\fontsize{10.0172}{1}\usefont{T1}{cmr}{m}{n}\selectfont\color{color_29791}1}
\put(359.7665,-687.8584){\fontsize{10.0172}{1}\usefont{T1}{cmr}{m}{n}\selectfont\color{color_29791}1}
\put(394.8067,-665.2997){\fontsize{10.0172}{1}\usefont{T1}{cmr}{m}{n}\selectfont\color{color_29791}22}
\put(331.9888,-674.8962){\fontsize{10.0172}{1}\usefont{T1}{cmr}{m}{n}\selectfont\color{color_29791}11}
\put(474.66,-670.6996){\fontsize{12.0209}{1}\usefont{T1}{cmr}{m}{n}\selectfont\color{color_29791}}
\put(356.46,-675.8596){\fontsize{20.0346}{1}\usefont{T1}{cmr}{m}{n}\selectfont\color{color_29791}}
\put(354.6,-687.8596){\fontsize{10.0172}{1}\usefont{T1}{cmr}{m}{n}\selectfont\color{color_29791}}
\put(404.76,-670.6996){\fontsize{12.0209}{1}\usefont{T1}{cmr}{m}{n}\selectfont\color{color_29791}dxpp;  }
\put(41.69276,-723.5555){\fontsize{12.0209}{1}\usefont{T1}{cmr}{m}{n}\selectfont\color{color_29791} }
\end{picture}
\begin{tikzpicture}[overlay]
\path(0pt,0pt);
\draw[color_29791,line width=0.998pt,line cap=round,line join=round]
(323.94pt, -720.5pt) -- (331.2pt, -720.5pt)
;
\draw[color_29791,line width=0.998pt,line cap=round,line join=round]
(376.5pt, -720.5pt) -- (387.78pt, -720.5pt)
;
\end{tikzpicture}
\begin{picture}(-5,0)(2.5,0)
\put(379.44,-733.0397){\fontsize{12.0209}{1}\usefont{T1}{cmr}{m}{n}\selectfont\color{color_29791}3}
\put(376.1343,-715.7657){\fontsize{12.0209}{1}\usefont{T1}{cmr}{m}{n}\selectfont\color{color_29791}11}
\put(358.2592,-723.5071){\fontsize{12.0209}{1}\usefont{T1}{cmr}{m}{n}\selectfont\color{color_29791}12}
\put(324.9012,-733.0517){\fontsize{12.0209}{1}\usefont{T1}{cmr}{m}{n}\selectfont\color{color_29791}3}
\put(324.6608,-715.7777){\fontsize{12.0209}{1}\usefont{T1}{cmr}{m}{n}\selectfont\color{color_29791}2}
\put(299.2846,-723.5191){\fontsize{12.0209}{1}\usefont{T1}{cmr}{m}{n}\selectfont\color{color_29791}1]1[),(}
\put(304.74,-718.0997){\fontsize{10.0172}{1}\usefont{T1}{cmr}{m}{n}\selectfont\color{color_29791}2}
\put(228.8998,-707.9022){\fontsize{10.0172}{1}\usefont{T1}{cmr}{m}{n}\selectfont\color{color_29791}1}
\put(231.965,-740.6584){\fontsize{10.0172}{1}\usefont{T1}{cmr}{m}{n}\selectfont\color{color_29791}1}
\put(267.7865,-718.0997){\fontsize{10.0172}{1}\usefont{T1}{cmr}{m}{n}\selectfont\color{color_29791}22}
\put(203.4661,-727.6962){\fontsize{10.0172}{1}\usefont{T1}{cmr}{m}{n}\selectfont\color{color_29791}22}
\put(366.48,-723.4997){\fontsize{12.0209}{1}\usefont{T1}{cmr}{m}{n}\selectfont\color{color_29791}}
\put(228.66,-728.6597){\fontsize{20.0346}{1}\usefont{T1}{cmr}{m}{n}\selectfont\color{color_29791}}
\put(226.8,-740.6597){\fontsize{10.0172}{1}\usefont{T1}{cmr}{m}{n}\selectfont\color{color_29791}}
\put(277.74,-723.4997){\fontsize{12.0209}{1}\usefont{T1}{cmr}{m}{n}\selectfont\color{color_29791}dxxpp}
\put(390.2436,-723.5598){\fontsize{12.0209}{1}\usefont{T1}{cmr}{m}{n}\selectfont\color{color_29791}. }
\end{picture}
\end{document}